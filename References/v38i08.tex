\documentclass[article,shortnames]{jss}
\usepackage{thumbpdf}

\graphicspath{{Figures/}}

\newcommand{\Exp}{\mathop{\mathrm{Exp}}}
\newcommand{\Rfunction}[1]{{\texttt{#1()}}}
\newcommand{\Robject}[1]{{\texttt{#1}}}
\newcommand{\Rpackage}[1]{{\pkg{#1}}}
\newcommand{\Rmethod}[1]{{\texttt{#1}}}
\newcommand{\Rfunarg}[1]{{\texttt{#1}}}
\newcommand{\Rclass}[1]{{`\texttt{#1}'}}

\author{Christopher H. Jackson\\Medical Research Council Biostatistics Unit}
\Plainauthor{Christopher H. Jackson}

\title{Multi-State Models for Panel Data: The \pkg{msm} Package for \proglang{R}}
\Plaintitle{Multi-State Models for Panel Data: The msm Package for R}
\Shorttitle{\pkg{msm}: Multi-State Models for Panel Data in \proglang{R}}

\Volume{38}
\Issue{8}
\Month{January}
\Year{2011}
\Submitdate{2009-07-21}
\Acceptdate{2010-08-18}

\Abstract{
  Panel data are observations of a continuous-time process at
  arbitrary times, for example, visits to a hospital to diagnose
  disease status.  Multi-state models for such data are generally
  based on the Markov assumption.  This article reviews the range of
  Markov models and their extensions which can be fitted to
  panel-observed data, and their implementation in the \pkg{msm}
  package for \proglang{R}.  Transition intensities may vary between
  individuals, or with piecewise-constant time-dependent covariates,
  giving an inhomogeneous Markov model.  Hidden Markov models can be
  used for multi-state processes which are misclassified or observed
  only through a noisy marker.  The package is intended to be
  straightforward to use, flexible and comprehensively documented.
  Worked examples are given of the use of \pkg{msm} to model chronic
  disease progression and screening.  Assessment of model fit, and
  potential future developments of the software, are also discussed. }

\Keywords{multi-state models, Markov models, panel data, \proglang{R}, \pkg{msm}}
\Plainkeywords{multi-state models, Markov models, panel data, R, msm}

\Address{
  Christopher Jackson\\
  Medical Research Council Biostatistics Unit\\
  Institute of Public Health\\
  Forvie Site, Robinson Way\\
  Cambridge, United Kingdom\\
  E-mail: \email{chris.jackson@mrc-bsu.cam.ac.uk}\\
  URL: \url{http://www.mrc-bsu.cam.ac.uk/}
}

\begin{document}


\section{Markov multi-state models for panel data}
\label{sec:intro}

\subsection{Definitions}

A multi-state model describes how an individual moves between a series
of states in continuous time. Suppose an individual is in state $S(t)$
at time $t$. The movement on the discrete state space $1,\ldots,R$ is
governed by \emph{transition intensities} $q_{rs}(t,z(t)):
r,s=1,\ldots,R$. These may depend on time $t$, or, more generally,
also on a set of individual-level or time-dependent explanatory
variables $z(t)$.  The intensity represents the instantaneous risk of
moving from state $r$ to state $s \neq r$:
\[
  q_{rs}(t, z(t))  =  \lim_{\delta t \rightarrow 0} \Prob (S(t+\delta t) = s | S(t) = r) / \delta t.
\]
The $q_{rs}$ form a $R \times R$ matrix $Q$ whose rows sum to zero, so
that the diagonal entries are defined by $q_{rr} = - \sum_{s \neq r}
q_{rs}$.  An example is the general model for disease
progression~(Figure~\ref{fig:disease}), in which individuals can
advance or recover between adjacent disease states, or die from any
state.

\subsection{Panel data}

The other articles in this issue focus on fitting multi-state models
of this type to \emph{continuously-observed} processes, where the
state $S_i(t)$ of each individual $i=1,\ldots,M$ is known at \emph{all
  times} $t$ in the study period.  Survival analysis is the simplest
such example, a two-state model where individuals remain alive until
an observed or censored time of death.

This article focuses on multi-state models for \emph{panel} data, in
which the state $S_i(t)$ is only known at a finite series of times $t
= (t_{i1}, \ldots, t_{in_i})$.  Fitting multi-state models to panel
data generally relies on the \emph{Markov} assumption, that future
evolution only depends on the current state.  That is, $q_{rs}(t,
z(t), \mathcal{F}_t)$ is independent of $\mathcal{F}_t$, the
observation history $\mathcal{F}_t$ of the process up to the time
preceding $t$.  See, for example, \citet{cox:miller} for a thorough
review of continuous-time Markov chain theory.  In a
\emph{time-homogeneous} Markov model, in which the $q_{rs}$ are also
independent of $t$, the sojourn time in each state $r$ is
exponentially-distributed with mean $-1/q_{rr}$.  The probability that
an individual in state $r$ moves next to state $s$ is $-q_{rs} /
q_{rr}$.

\subsection[The msm package]{The \pkg{msm} package}

This article describes the \pkg{msm} package for \proglang{R}
\citep{R}, available from \url{http://CRAN.R-project.org/package=msm}.
\pkg{msm} can be used to fit a Markov model with any number of states
and any pattern of transitions to panel data, and includes several
extensions such as hidden Markov models and models whose transition
intensities vary with individual-specific or time-varying covariates.
\pkg{msm} was motivated by studies of chronic diseases in medicine,
and is frequently used in this area
\citep{jackson03:_multis_markov,my:cav,my:heron,sweeting2006epr,buter2008das,eirik:msm},
but it has been widely used in other fields, for example
geology~\citep{aspinall}, zoology~\citep{gautrais2007asm} and
econometrics~\citep{rummel}.

\begin{figure}[htbp]
  \centering
  \vskip 1cm
  \scalebox{1.0}{\includegraphics{general}}
    \[
    Q = \left(
      \begin{array}{llllll}
        q_{11} & q_{12} & 0 & 0 & \ldots & q_{1R}\\
        q_{21} & q_{22} & q_{23} & 0 & \ldots & q_{2R}\\
        0 & q_{32} & q_{33} & q_{34} & \ddots & q_{3R}\\
        0 & 0 & q_{43} & q_{44} & \ddots & q_{4R}\\
        \vdots & \vdots & \ddots & \ddots & \ddots & \vdots\\
        0 & 0 & 0 & 0 & \ldots & 0\\
      \end{array}
    \right )
    \]
  \caption{\label{fig:disease}General model for disease progression. Individuals advance between adjacent stages of disease severity, and optionally recover to an adjacent less severe state or die from any state.}
  \end{figure}


\subsection{Likelihood for panel data}
\label{sec:lik}

The Markov model for panel data was first described
by~\citet{kalbfleisch:lawless} and~\citet{kay:mark}.  The likelihood
for this basic model, used in \pkg{msm}, is calculated from the
\emph{transition probability matrix} $P(u, t+u)$.  The $(r,s)$ entry
of $P(u, t+u)$, $p_{rs}(u,t+u)$, is the probability of being in state
$s$ at a time $t+u$, given the state at time $u$ is $r$.  $P(u,t+u)$
is calculated in terms of $Q$ using the Kolmogorov differential
equations \citep{cox:miller}.  If the transition intensity matrix $Q$
is constant over the interval $(u,t+u)$, as in a time-homogeneous
process, then $P(u, t+u) = P(t)$ and the equations are solved by the
matrix exponential of $Q$ scaled by the time interval,
\[
  P(t) = \Exp(tQ).
\]
The matrix exponential $\Exp()$ is notoriously difficult to calculate
reliably, as discussed by \citet{matrixexp}.  It is defined by the
same ``power series'' $\Exp(X) = 1 + X^2/2! + X^3/3! + ...$ as the
scalar exponential, except that each term $X^k$ in the series is
defined by matrix products, not element-wise scalar multiplication.
For simpler models, an analytic expression for each element of $P(t)$
can be calculated in terms of entries of $Q$ by hand or by using
symbolic algebra software.  Otherwise, \pkg{msm} uses eigensystem
decomposition, or, if there are repeated eigenvalues, the method of
Pad\'e approximants \citep{matrixexp}.

The full likelihood is then the product of probabilities of transition
between observed states, over all individuals $i$ and observation
times $j$:
\begin{equation}
  \label{eq:multi:lik:contrib}
  L(Q) = \prod_i L_i = \prod_{i,j} L_{i, j} =  \prod_{i,j} p_{S(t_{ij})S(t_{i,j+1})}(t_{i,j+1} - t_{ij}).
\end{equation}
Each component $L_{i,j}$ is the entry of the transition matrix $P(t)$
at the $S(t_{ij})$th row and $S(t_{i,j+1})$th column, evaluated at $t
= t_{i,j+1} - t_{ij}$.  The likelihood $L(Q)$ is maximized in terms of
$\log(q_{rs})$ to compute the estimates of $q_{rs}$, using standard
optimization algorithms, as implemented in the \code{optim} function
in \proglang{R}. Standard errors are computed from the Hessian at the
optimum.  Some of these optimization algorithms make use of the
derivatives of the likelihood, which were given
by~\citet{kalbfleisch:lawless}.

The likelihood~(\ref{eq:multi:lik:contrib}) for this and all models in
\pkg{msm} assumes that the sampling times are ignorable.  That is, the
fact that a particular observation is made at a certain time does not
implicitly give information about the value of that observation.
Sampling times are ignorable if they are fixed in advance, or
otherwise chosen independently of the outcome of the process.
\citet{gruger:valid} also showed that the sampling times are ignorable
under a \emph{``doctor's care''} sampling scheme, where the next
observation time (such as a visit to a doctor) is chosen on the basis
of the current state.  Basing the \emph{current} observation time on
the current state would be a non-ignorable sampling scheme.  To avoid
bias, non-ignorable sampling times should be modelled as part of the
likelihood~\citep{sweeting:inform:msm:jss}.

\subsection{Likelihood under alternative observation schemes}
\label{sec:altobs}

\subsubsection{Exact death times}
\label{sec:exactdeath}
In observational studies of chronic diseases, it is common that the
time of death is known, but the state immediately before death is
unknown. If $S(t_{i,j+1}) = D$ is such a death state, then the
contribution to the likelihood at this time is summed over the unknown
state $m$ at the instant before death:
\[
  L_{i, j} =  \sum_{m \neq D} p_{S(t_{ij}),m}(t_{i,j+1} - t_{ij}) q_{m, D}
\]

\subsubsection{Continuously-observed processes}
\pkg{msm} allows Markov models to be fitted to processes which are
continuously-observed.  However, the assumption of exponential sojourn
times inherent in Markov models is restrictive, and more flexible
models can be fitted to such data with other software. For example,
proportional hazards models with non-parametric baseline intensities
can be fitted using the \pkg{mstate} package \citep{i07,mstate:jss}

Generally, \pkg{msm} allows a dataset to be an arbitrary mixture of
observations such that states are panel-observed,
continuously-observed, or ``exact death times''.


\section[Using msm for a basic Markov model]{Using \pkg{msm} for a basic Markov model}
\label{sec:basic}

The package is illustrated with a set of data from monitoring heart
transplant recipients, which is provided with \pkg{msm}.
\citet{my:cav} studied the progression of coronary allograft
vasculopathy (CAV), a post-transplant deterioration of the arterial
walls, using these data.  The dataset can be made available to the
current \proglang{R} session using the command \code{data("cav")}.  30
observations from 8 individuals with missing primary diagnosis (reason
for transplantation, variable \code{pdiag}) are dropped from the data,
giving a dataset with 2816 state observations from 614 individuals.

\begin{CodeInput}
R> library("msm")
R> data("cav")
R> cav <- cav[!is.na(cav$pdiag),]
\end{CodeInput}
%$

\subsection{Format of data}

Approximately each year after transplant, each patient has an
angiogram, at which CAV can be diagnosed. The result of the test is in
the variable \Robject{state}, with possible values:
\begin{itemize}
\item 1, representing no CAV.
\item 2, representing mild/moderate CAV.
\item 3, representing severe CAV.
\item 4, recorded at the date of death.
\end{itemize}
\Robject{years} gives the time of the test in years since the heart
transplant.

Data are supplied to \pkg{msm} as a series of observations, grouped by
patient. This should be a data frame with variables indicating the
observed state of the process (\code{state} in the CAV data) and the
time of the observation (\code{years} in the CAV data) If the data
come from more than one individual, then a subject identification
variable (\code{PTNUM} in the CAV data) must also be supplied.  This
does not need to be numeric, but observations from the same subject
must be adjacent in the dataset, and observations must be ordered by
time within subjects.  The first eleven rows of the data \code{cav}
give the observation series from the first two patients.  Other
variables are either individual-specific or time-dependent covariates
(see Section~\ref{sec:covs}).

\begin{CodeChunk}
  \begin{CodeInput}
R> cav[1:11,]
  \end{CodeInput}
  \begin{CodeOutput}
    PTNUM      age    years dage sex pdiag cumrej state firstobs
1  100002 52.49589 0.000000   21   0   IHD      0     1        1
2  100002 53.49863 1.002740   21   0   IHD      2     1        0
3  100002 54.49863 2.002740   21   0   IHD      2     2        0
4  100002 55.58904 3.093151   21   0   IHD      2     2        0
5  100002 56.49589 4.000000   21   0   IHD      3     2        0
6  100002 57.49315 4.997260   21   0   IHD      3     3        0
7  100002 58.35068 5.854795   21   0   IHD      3     4        0
8  100003 29.50685 0.000000   17   0   IHD      0     1        1
9  100003 30.69589 1.189041   17   0   IHD      1     1        0
10 100003 31.51507 2.008219   17   0   IHD      1     3        0
11 100003 32.49863 2.991781   17   0   IHD      2     4        0
  \end{CodeOutput}
\end{CodeChunk}

Multi-state data can be summarized by counting, for each $r$ and $s$,
the number of times an observation of state $r$ was followed by state
$s$.  This is implemented in the function \code{statetable.msm()}.  In
this example, an observation of severe CAV (state 3) was followed by a
less severe state (states 1--2) on only 17 occasions.

\begin{CodeChunk}
\begin{CodeInput}
R> statetable.msm(state, PTNUM, data = cav)
\end{CodeInput}
\begin{CodeOutput}
    to
from    1    2    3    4
   1 1348  203   44  147
   2   46  134   54   47
   3    4   13  107   55
\end{CodeOutput}
\end{CodeChunk}

\subsection{Specifying the Markov model and initial values}
We assume that the patient can advance or recover from consecutive
states while alive, and die from any state, as in
Figure~\ref{fig:disease} with $R=4$ states, giving a transition
intensity matrix of
\[
Q = \left(
  \begin{array}{llll}
    -(q_{12} + q_{14}) & q_{12} &  0     & q_{14}\\
    q_{21} & -(q_{21}+q_{23}+q_{24}) & q_{23} & q_{24}\\
      0    & q_{32} & -(q_{32}+q_{34}) & q_{34}\\
      0    &   0    &   0    &   0   \\
  \end{array}
\right )
\]
As Section~\ref{sec:hmm} will explain, this model is not strictly
medically realistic, but we fit it here for illustration.  Note that
this matrix represents transitions \emph{in an instant} rather than
the transitions \emph{over an interval} summarized by
\code{statetable.msm} -- the 4 individuals who moved from state 3 to
state 1 in successive observations are assumed to have travelled via
state 2, therefore $q_{31} = 0$ but $q_{32},q_{21} \neq 0$.

To tell \pkg{msm} what the allowed transitions of our model are, we
define a matrix \code{twoway4.q} of the same size as $Q$, containing
zeroes in the off-diagonal positions where the entries of $Q$ are
zero.  All other off-diagonal positions contain an \emph{initial
  value} for the corresponding transition intensity.  Any diagonal
entries $q_{rr}$ supplied are ignored, as these are constrained to be
minus the sum of all the other entries in the row.  The rows and
columns of \code{twoway4.q} are given informative names which will be
used when presenting the estimates.

\begin{CodeInput}
R> twoway4.q <- rbind(c(0, 0.25, 0, 0.25), c(0.166, 0, 0.166, 0.166),
+    c(0, 0.25, 0, 0.25), c(0, 0, 0, 0))
R> rownames(twoway4.q) <- colnames(twoway4.q) <- c("Well", "Mild",
+    "Severe", "Death")
\end{CodeInput}

In this example, the initial values represent a guess that the mean
period in each state before moving to the next is about 2 years
($q_{rr}=-0.5$) and there is an equal probability of progression,
recovery or death ($q_{rr}=-\sum_{s \neq r} q_{rs}$).  Alternatively,
by supplying the option \code{gen.inits=TRUE} to \code{msm()}, the
initial values for non-zero entries of $Q$ can be set to the maximum
likelihood estimates under the assumption that transitions take place
only at the observation times.

\subsection[Running msm and interpreting results]{Running \pkg{msm} and interpreting results}
\label{sec:runningmsm}

The maximum likelihood estimate of $Q$ is computed by the \code{msm()}
function, as below, starting from the supplied initial values. The
argument \code{death=4} indicates that entry times into state 4 are
observed exactly but the state on the instant before is unknown
(Section~\ref{sec:altobs}).  The optimization in this example takes
about 20 seconds on a typical current computer. Printing the object
\code{cav.msm} returned by \code{msm()} displays the estimated
transition intensity matrix with 95\% confidence intervals.  We see
patients are about three times as likely to develop CAV than die
without CAV (first row).  After onset of mild disease, progression to
severe CAV is about 50\% more likely than recovery, and death from the
severe disease state is rapid (mean of 1 / 0.41 = 2.4 years in state
3).

\begin{CodeChunk}
\begin{CodeInput}
R> cav.msm <- msm(state ~ years, subject = PTNUM, data = cav,
+    qmatrix = twoway4.q, death = 4)
R> cav.msm
\end{CodeInput}
\begin{CodeOutput}
Call:
msm(formula = state ~ years, subject = PTNUM, data = cav,
    qmatrix = twoway4.q, death = 4)

Maximum likelihood estimates:
Transition intensity matrix

       Well                     Mild
Well   -0.1682 (-0.188,-0.1505) 0.1276 (0.111,0.1467)
Mild   0.2264 (0.1692,0.303)    -0.618 (-0.7195,-0.5309)
Severe 0                        0.1226 (0.07308,0.2056)
Death  0                        0
       Severe                    Death
Well   0                         0.04057 (0.03227,0.051)
Mild   0.3375 (0.2713,0.4199)    0.05405 (0.02233,0.1308)
Severe -0.4144 (-0.5245,-0.3275) 0.2919 (0.2274,0.3746)
Death  0                         0

-2 * log-likelihood:  3945.363
\end{CodeOutput}
\end{CodeChunk}

To display the fitted transition probability matrix $P(t)$ over an
interval of $t=1$ year, the function \code{pmatrix.msm()} is used.
This suggests a 9\%, 15\% and 4\% probability that in one year's time,
an individual currently free of CAV will have mild CAV, severe CAV or
be dead, respectively.  The option \code{ci="normal"} computes a
confidence interval for $P(t)$ by repeated sampling from the
asymptotic normal distribution of the maximum likelihood estimates of
the $\log(q_{rs})$.  The output below is based on the default 1000
samples, and has converged to within 2 significant figures.
Alternatively, intervals can be computed using nonparametric bootstrap
resampling (\code{ci="boot"}). The dataset of $\sum_{i=1}^M n_i$
serially-correlated state observations from $M$ individuals is
rearranged as a dataset of $\sum_{i=1}^M (n_i-1)$ independent
transitions between pairs of states.  Bootstrap datasets of
transitions are drawn with replacement and the model refitted
repeatedly to estimate the sampling uncertainty surrounding the
estimates.  This method is more accurate but much slower due to the
need to refit the model for each resample.

\begin{CodeChunk}
\begin{CodeInput}
R> pmatrix.msm(cav.msm, t = 1, ci = "normal")
\end{CodeInput}

\begin{CodeOutput}
       Well                        Mild
Well   0.8558 (0.8421,0.8691)      0.08785 (0.07671,0.09852)
Mild   0.1559 (0.1194,0.2027)      0.5602 (0.5035,0.6012)
Severe 0.009393 (0.005273,0.01624) 0.07416 (0.04487,0.1198)
Death  0                           0
       Severe                    Death
Well   0.01458 (0.01148,0.01824) 0.04175 (0.03482,0.05131)
Mild   0.2042 (0.1678,0.2445)    0.07974 (0.06067,0.1267)
Severe 0.6736 (0.6035,0.7275)    0.2429 (0.197,0.2952)
Death  0                         1 (1,1)
\end{CodeOutput}
\end{CodeChunk}

\subsection{Controlling numerical optimization}
The optimization may occasionally converge to a local rather than a
global maximum of the likelihood surface.  Therefore to ensure that
the global maximum has been found, it is recommended to run
\code{msm()} with diverse sets of initial values.  However, if values
too far from the optimum are chosen then the algorithm may not
converge.  To improve convergence, the optimization in \code{msm()}
can be fine-tuned using all the options available to the \proglang{R}
function \code{optim()}.  For example, the number of iterations can be
increased with \code{maxit}, and the log-likelihood can be rescaled
during optimization (\code{fnscale}).

But if over-complex models are applied with insufficient data, then
the parameters of the model will not be identifiable.  The
\code{fixedpars} option to \code{msm()} is useful for profiling
likelihoods.  This allows any parameters to be fixed at their initial
values.  The model must of course be realistic.  In Markov models for
panel data, it is not usually feasible to estimate a model where
\emph{instantaneous} transitions are allowed between every pair of
states. For example, in chronic disease applications, transitions are
generally only plausible between ``adjacent'' states of a disease -- a
patient who is observed as ``well'' at $t_j$, and ``severe'' at
$t_{j+1}$ must have gone through ``mild'' in the interval
($t_j,t_{j+1}$).


\section{Markov models with covariates}
\label{sec:covs}

\subsection{Individual-level covariates}
\label{sec:indivcovs}

The effect of a vector of explanatory variables $\mathbf{z}_{ij}$ on the transition intensity for individual $i$ at time $j$ is
modelled using proportional intensities, replacing $q_{rs}$ with
\[q_{rs}(z_{ij}) = q_{rs}^{(0)} \exp(\mbox{\boldmath$\beta$\unboldmath}_{rs}^\top \mathbf{z}_{ij}).\]
The likelihood is then maximized over the $q_{rs}^{(0)}$ and $\mbox{\boldmath$\beta$\unboldmath}_{rs}$.

In the CAV example, the age of the heart transplant donor (variable
\code{dage}) and the primary diagnosis, or reason for transplantation
(variable \code{pdiag}), are suggested to affect the rate of onset and
progression of CAV.  We fit a model in which the intensities are
different according to donor age and a primary diagnosis of ischaemic
heart disease (IHD), after creating a binary variable \code{ihd}
representing IHD from the categorical \code{pdiag}.  A ``formula'' in
standard \proglang{R} linear modelling syntax, \code{~ dage + ihd}, is
supplied as the \code{covariates} argument to \code{msm()}.  To
facilitate convergence, the ``BFGS'' quasi-Newton optimization
algorithm is used (see the documentation for the \proglang{R} function
\code{optim()}), and the maximum number of iterations is increased to
10000.  The -2$\times$ log-likelihood is also divided by 4000, since
it takes values around 4000 for plausible ranges of the parameters.
This ensures that optimization takes place on an approximate unit
scale, to avoid numerical overflow or underflow.

\begin{CodeInput}
R> ihd <- as.numeric(cav[, "pdiag"] == "IHD")
R> cav.cov.msm <- msm(state ~ years, subject = PTNUM, data = cav,
+    covariates = ~ dage + ihd, qmatrix = twoway4.q, death = 4,
+    method = "BFGS", control = list(fnscale = 4000, maxit = 10000))
\end{CodeInput}

Instead of printing the fitted model object \code{cav.cov.msm}, which
shows the relatively uninformative baseline intensities $q_{rs}^{(0)}$
and log hazard ratios $\mbox{\boldmath$\beta$\unboldmath}_{rs}$, we
use the function \code{hazard.msm()} to display hazard ratios
$\exp(\mbox{\boldmath$\beta$\unboldmath}_{rs})$ for each covariate on
each transition with 95\% confidence intervals.  A primary diagnosis
of IHD is associated with a 56\% increase in the hazard of CAV onset,
and 1 year of donor age is associated with a 2\% greater risk of CAV
onset and a 4\% greater risk of death without CAV.

\begin{CodeChunk}
\begin{CodeInput}
R> hazard.msm(cav.cov.msm)
\end{CodeInput}
\begin{CodeOutput}
$dage
                      HR         L        U
Well - Mild    1.0192556 1.0068692 1.031794
Well - Death   1.0381769 1.0180960 1.058654
Mild - Well    0.9981484 0.9725701 1.024399
Mild - Severe  0.9856091 0.9674640 1.004095
Mild - Death   0.9320659 0.8448829 1.028245
Severe - Mild  0.9976255 0.9476498 1.050237
Severe - Death 0.9884293 0.9648293 1.012607

$ihd
                      HR         L         U
Well - Mild    1.5647641 1.1793343  2.076160
Well - Death   1.3044011 0.8207672  2.073014
Mild - Well    0.9372774 0.5193818  1.691413
Mild - Severe  0.9578794 0.6126934  1.497540
Mild - Death   1.7858347 0.2298841 13.873100
Severe - Mild  0.7669038 0.2706515  2.173058
Severe - Death 0.7572325 0.4562969  1.256641
\end{CodeOutput}
\end{CodeChunk}

We can also use \code{qmatrix.msm()} to calculate the transition
intensity matrix for specified covariate values as follows, in this
case, a donor age of 50 years old and a primary diagnosis of IHD.
Compared with the fitted intensities for the ``average'' person from the
model without covariates (Section~\ref{sec:runningmsm}), we see an
approximately doubled risk of CAV onset and death without CAV.  (The
average donor is 30 years old and about half of heart transplants are
due to IHD).

\begin{CodeChunk}
\begin{CodeInput}
R> qmatrix.msm(cav.cov.msm, covariates = list(dage = 50, ihd = 1))
\end{CodeInput}
\begin{CodeOutput}
       Well                      Mild
Well   -0.3438 (-0.4388,-0.2693) 0.2467 (0.1825,0.3335)
Mild   0.2201 (0.1153,0.4201)    -0.4811 (-0.6876,-0.3366)
Severe 0                         0.112 (0.03684,0.3404)
Death  0                         0
       Severe                    Death
Well   0                         0.09707 (0.06499,0.145)
Mild   0.2485 (0.1587,0.3891)    0.01257 (0.0007542,0.2096)
Severe -0.3233 (-0.5476,-0.1909) 0.2113 (0.1215,0.3676)
Death  0                         0
\end{CodeOutput}
\end{CodeChunk}

\subsection{Model comparison}

Likelihood ratio tests between nested models fitted in \pkg{msm} can
be performed conveniently using the function \code{lrtest.msm}.
Comparing a likelihood ratio statistic of 59 to a $\chi^2$
distribution with 14 degrees of freedom shows that the model with
covariates (\code{cav.cov.msm}) fits significantly better than the
model without covariates (\code{cav.msm}).

\begin{CodeChunk}
\begin{CodeInput}
R> lrtest.msm(cav.msm, cav.cov.msm)
\end{CodeInput}
\begin{CodeOutput}
            -2 log LR df            p
cav.cov.msm  58.5785 14 2.079552e-07
\end{CodeOutput}
\end{CodeChunk}

Covariate effects may be constrained to equal between different
intensities, using the \code{constraint} argument to \code{msm()}.
For example, in a disease progression model, the effect of a covariate
on all progression rates may be equal.  \code{constraint} is a list of
vectors, one for each covariate. In the model \code{cav.cov2.msm}
fitted below, \code{dage = c(1, 2, 3, 1, 2, 4, 2)} indicates that the effect
of \code{dage} on the 1st and 4th intensities are constrained to be
equal, as is the effect on the the 2nd, 5th and 7th intensities.  The
parameters are assumed to be ordered by reading across the rows of the
transition matrix, starting at the first row: ($q_{12}, q_{14},
q_{21}, q_{23}, q_{24}, q_{32}, q_{34}$), so that in the model
\code{cav.cov2.msm}, the effect on the CAV onset rate $q_{12}$ equals
the effect on the CAV progression rate $q_{23}$, and the effects on
all death rates $q_{14},q_{24},q_{34}$ are constrained to be equal.
However, a likelihood ratio test indicates that the bigger model
\code{cav.cov.msm} without constraints fits significantly better.

\begin{CodeChunk}
\begin{CodeInput}
R> cav.cov2.msm <- msm(state ~ years, subject = PTNUM, data = cav,
+    covariates = ~ dage + ihd, constraint = list(dage =
+    c(1, 2, 3, 1, 2, 4, 2), ihd = c(1, 2, 3, 1, 2, 4, 2)),
+    qmatrix = twoway4.q, death = 4)
R> lrtest.msm(cav.cov2.msm, cav.cov.msm)
\end{CodeInput}
\begin{CodeOutput}
            -2 log LR df            p
cav.cov.msm  60.10682  6 4.281631e-11
\end{CodeOutput}
\end{CodeChunk}

Some intensities may not be influenced by covariates at all.  In
\pkg{msm}, models in which covariates affect some intensities, but not
others, can be specified by fixing certain covariate effects at their
default initial values of zero, by instructing the optimizer not to
optimize over those parameters using the \code{fixedpars} argument to
\code{msm()}. See the package help for further details.


\subsection{Time-inhomogeneous models}
\label{sec:inhomog}

In general, the transition probability matrix $P(u, t+u)$, hence the
likelihood for panel data, cannot be calculated in closed form if $Q$
varies over the interval $(u, t+u)$.  An exception is if $Q$ is
piecewise-constant.  The effect of time-dependent variables, including
time itself, on the transition intensities can be modelled in
\pkg{msm} under this assumption.  For example, suppose a covariate
varies continuously through time, but is only observed at the same
times as the state of the Markov process.  The approximate effect of
that covariate can be estimated assuming that it is constant in
between the times that it is observed, so that $P(u, t+u)=P(t)$.  More
generally, time-inhomogeneous Markov models can be constructed in
which piecewise-constant covariates change at times other than
$(t_{i1}, \ldots, t_{in_i})$.  This is accomplished by summing the
likelihood over the unknown observed state at the times when the
covariates change (Equation~\ref{eq:cens}, Section~\ref{sec:censoring}).

\pkg{msm} provides a convenient facility for constructing
time-inhomogeneous models in which intensities change at the same
times for every individual.  A vector of change points is specified in
the \code{pci} argument to \code{msm()}.  The following command fits
an inhomogeneous model to the CAV data in which all intensities change
5 years after transplantation.  This constructs a model with a single
binary covariate called \code{timeperiod}, a \emph{factor} in
\proglang{R}, with levels \code{(-Inf, 5]} (the baseline) representing
the first time period, and \code{[5, Inf)}, representing the second
time period.  A likelihood ratio test against the time-homogeneous
model suggests significant time-inhomogeneity.  The estimated hazard
ratios from this fitted model show an increased onset rate of mild CAV
in the second period, though no significant time effect on other
transitions.  There is weak information about the effect of time on
the death rate from mild CAV.

\begin{CodeChunk}
  \begin{CodeInput}
R> cav.pci.msm <- msm(state ~ years, subject = PTNUM, data = cav,
+    qmatrix = twoway4.q, death = 4, pci = 5, method = "BFGS")
R> lrtest.msm(cav.msm, cav.pci.msm)
  \end{CodeInput}
  \begin{CodeOutput}
            -2 log LR df            p
cav.pci.msm  49.24128  7 2.034911e-08
  \end{CodeOutput}
  \begin{CodeInput}
R> hazard.msm(cav.pci.msm)
  \end{CodeInput}
  \begin{CodeOutput}
$`timeperiod[5,Inf)`
                       HR         L           U
Well - Mild     2.2080439 1.6418440    2.969501
Well - Death    0.6714820 0.2472622    1.823522
Mild - Well     0.6634596 0.3581871    1.228907
Mild - Severe   0.9165669 0.5747890    1.461571
Mild - Death   12.9314664 0.1392106 1201.221729
Severe - Mild   1.4253788 0.4753785    4.273867
Severe - Death  1.6828792 0.8470715    3.343381
  \end{CodeOutput}
\end{CodeChunk}
%$

Time-dependent intensities in \pkg{msm} are restricted to
piecewise-constant models.  More flexible alternatives are discussed
in Section~\ref{sec:limitations}.

\subsection{Censored states}
\label{sec:censoring}

In the CAV example, some patients were known to be alive but in an
unknown disease state at the end of the study.  We say that the
disease state is \emph{censored}, meaning that the exact value is
unknown, but known to be in a certain set.  Unlike in survival
analysis, here it is the state, not the event time, which is censored.
If the patient were alive at the end of the study but with a known
state, then the standard likelihood~(\ref{eq:multi:lik:contrib}) would
apply.

\pkg{msm} allows the state observation at any time to be censored,
that is, known only to be in an arbitrary subset of the state space.
Suppose the $1, 2, \ldots n_i$th observations from individual $i$ are
known only to be in the sets $C_1, C_2, \ldots, C_{n_i}$ respectively.
The likelihood for this individual is a sum of the likelihoods of all
possible paths through the unobserved states.

\begin{equation}
  \label{eq:cens}
  L_i = \sum_{s_{n_i} \in C_{n_i}} \ldots \sum_{s_2 \in C_2} \sum_{s_1 \in C_1} p_{s_1 s_2}(t_2 - t_1) p_{s_2 s_3} (t_3 - t_2) \ldots p_{s_{n_i-1} s_{n_i}} (t_{n_i} - t_{n_i-1})
\end{equation}

This likelihood is used in \pkg{msm} to fit general time-inhomogeneous
models with piecewise-constant intensities, as described in
Section~\ref{sec:inhomog}, where the state is not observed at times
when the intensities change.

Suppose the variable \code{state} in the data \code{cav} were to
contain observations coded 99 on occasions where the patient is alive
but in an unknown state, which could be state 1, 2 or 3.  The standard
Markov model could be fitted to such data using the \code{censor} and
\code{censor.states} options to \code{msm()}, as follows.

\begin{CodeInput}
R> cav.msm <- msm(state ~ years, subject = PTNUM, data = cav,
+    qmatrix = twoway4.q, death = TRUE, censor = 99,
+    censor.states = c(1, 2, 3))
\end{CodeInput}



\section{Hidden Markov models}
\label{sec:hmm}

In a {\em hidden Markov model} (HMM), the states of the Markov chain
are not observed. The observed data $y_{ij}$ are governed by some
probability distribution conditionally on the unobserved state
$S_{ij}$.  This class of model is commonly used in areas such as
speech and signal processing \citep{juang:rabiner} and the analysis of
biological sequence data \citep{biolog:seq}, with a discrete-time
underlying Markov chain.  Applications of HMMs in medicine, where
continuous-time processes are usually more suitable, include
\citet{sattlong,bureau:sim,JacksonS02,jackson03:_multis_markov}.
These models can represent chronic staged diseases which can only be
diagnosed by an error-prone marker.

\subsection{Likelihood}

The \pkg{msm} package can fit continuous-time hidden Markov models to
panel-observed data with a variety of distributions for the outcome
conditionally on the hidden state.  HMMs are fitted in \pkg{msm} by
direct maximization of the likelihood, as in \citet{sattlong}, though
\citet{bureau:jss} describe an alternative EM algorithm for fitting
the same class of models.  The contribution of individual $i$ to the
likelihood is
\begin{eqnarray}
  \label{eq:multi:hiddencontrib}
  L_i     & = & \Prob(y_{i1}, \ldots, y_{in_i})\\
  & = & \sum \Prob(y_{i1}, \ldots, y_{in_i} | S_{i1}, \ldots, S_{in_i})
  \Prob(S_{i1}, \ldots, S_{in_i}) \nonumber
\end{eqnarray}
where the sum is taken over all possible paths of underlying states
$S_{i1}, \ldots, S_{in_i}$.  Assume that the observed states are
conditionally independent given the values of the underlying states.
Also assume the Markov property, $\Prob(S_{ij}|S_{i,j-1}, \ldots, S_{i1})
= \Prob(S_{ij}|S_{i,j-1})$.  Then the contribution $L_i$ can be written
as a product of matrices, as follows. To derive this matrix product,
decompose the overall sum in Equation~\ref{eq:multi:hiddencontrib}
into sums over each underlying state. The sum is accumulated over the
unknown first state, the unknown second state, and so on until the
unknown final state:
\[
L_i  =  \sum_{S_{i1}} \Prob(y_{i1}|S_{i1})\Prob(S_{i1}) \sum_{S_{i2}} \Prob(y_{i2}|S_{i2})\Prob(S_{i2}|S_{i1}) \ldots \sum_{S_{in_i}} \Prob(y_{in_i}|S_{in_i}) \Prob(S_{in_i}|S_{in_{i-1}})
\]
where $\Prob(y_{ij}|S_{ij})$ is the probability density of the outcome
conditional on the hidden state (also called the ``emission''
distribution), and $\Prob(S_{ij}|S_{i,j-1})$ is the transition
probability of the hidden Markov chain, calculated as in
Section~\ref{sec:lik}.

\pkg{msm} allows most common distributions to be used as HMM outcome
models.  The modular design of \pkg{msm} allows new outcome
distributions to be added easily, as described in the package
documentation.  These must be univariate, and \pkg{msm} is restricted
to situations where only one observation is made conditionally on an
underlying Markov process.

In practice, the outcome distribution may vary between individuals and
through time, as well as with the hidden state. \pkg{msm} allows one
\emph{location} parameter for each class of outcome distribution to
depend on covariates, for example, a linear model for the mean of a
normal outcome distribution.  The transition rates of the hidden
Markov chain may also vary with covariates, just as for non-hidden
Markov models (Section~\ref{sec:indivcovs}).

The distribution $\Prob(S_{i1})$ of the initial state may be estimated
from the data, or fixed at plausible values.  This distribution may
also depend on covariates through a multinomial logistic regression.


\subsection{Application of hidden Markov models: FEV$_1$ after lung transplants}
\label{sec:fev}

A dataset of repeated measurements of FEV$_1$, forced expiratory
volume in 1 second, in recipients of lung transplants
\citep{JacksonS02} is provided with \pkg{msm} as \code{data("fev")}.
FEV$_1$ measurements are used to diagnose bronchiolitis obliterans
syndrome (BOS), a chronic deterioration in lung function.  FEV$_1$ is
measured as a percentage of a baseline value for each individual,
determined in the first six months after transplant, and defined to be
100\% baseline at six months.  Figure~\ref{fig:fev} shows a series of
FEV$_1$ measurements from a typical patient from this dataset.  BOS is
modelled as a staged disease, with stages defined by
\begin{itemize}
\item No BOS ($\geq$ 80\% baseline FEV$_1$).
\item Mild BOS (sustained drop below $<$ 80\% baseline FEV$_1$).
\item Moderate BOS (sustained drop below $<$ 65\% baseline FEV$_1$).
\item Severe BOS (sustained drop below $<$ 50\% baseline FEV$_1$).
\item Death.
\end{itemize}
As FEV$_1$ is subject to high short-term variability due to acute
events and natural fluctuations, the exact state at each observation
time is difficult to determine, making it difficult to model the
natural history of BOS as defined.  Instead, we represent the BOS
progression by a hidden Markov model for FEV$_1$, conditionally on
underlying BOS states.  Discrete states are considered to be an
appropriate alternative to representing the underlying disease status
as continuous, as the onset of BOS is often sudden.

\begin{figure}[t!]
  \centering
  \scalebox{1.0}{\includegraphics{fev}}
  \caption{Measurements of lung function (FEV$_1$) from a lung transplant recipient and fitted BOS states from a hidden Markov model.\label{fig:fev}}
\end{figure}

Here we describe a three-state ``illness-death'' hidden Markov model,
with states representing no BOS, BOS and death, and a transition
intensity matrix of
\[
Q = \left(
  \begin{array}{llll}
    -q_{12} & q_{12} &  0  \\
    0  & -q_{23} & q_{23}\\
    0    &   0    &   0   \\
  \end{array}
\right )
\]
The distribution of percentage of baseline FEV$_1$ is Normal$(\mu_1,
\sigma^2_1)$ in state 1 and Normal$(\mu_2, \sigma^2_2)$ in state 2.
State 3, representing death, is observed without error and is given a
label of 999 in the data. The death time is known exactly.  More
sophisticated four and five-state models for the FEV$_1$ data, using
outcome distributions which separate measurement error and natural
variation in the response, are described by~\citet{JacksonS02}.


\subsection[Fitting hidden Markov models with msm]{Fitting hidden Markov models with \pkg{msm}}
\label{sec:fittinghmm}

To fit this hidden Markov model using the \code{msm()} function, the
argument \code{hmodel} is used.  This is a list of objects
representing the outcome distribution for each state, returned by
\emph{constructor} functions.  Each constructor function has arguments
giving initial values for the parameters of the outcome distribution.
In this example, \code{hmmNorm(mean = 100, sd = 16)} indicates initial
values of 100 for $\mu_1$ and 16 for $\sigma_1$.  \code{hmmIdent(999)}
represents the identity distribution, in other words, state 3 is
observed without error, and is indicated by a value of 999 in the
data.  Initial values for the Markov transition intensities are given
in an object called \code{three.q}, used as the \code{qmatrix}
argument to \code{msm()} as before.

The FEV$_1$ values, conditional on the BOS state, are assumed to be
affected by a time-dependent covariate indicating whether the patient
suffered acute infections or rejection episodes within 14 days of the
observation.  To model this covariate effect we use the\linebreak
\code{hcovariates} argument to \code{msm()}.  This takes a list of
linear model formulae, which are used for the location parameter of
the respective outcome distribution.  In this case, the means $\mu_1$
and $\mu_2$ of the normal distribution have a linear model with a
single binary covariate \code{acute}.  The \code{hconstraint}
statement (analogous to \code{constraint}) indicates that the effect
of acute events on $\mu_1$ and $\mu_2$ is constrained to be the same.
No covariates are assumed to affect the transition rates $Q$ in this
example, but \code{covariates} and \code{constraint} arguments could
be included for this purpose just as in Section~\ref{sec:indivcovs}.

\begin{CodeChunk}
  \begin{CodeInput}
R> data("fev")
R> three.q <- rbind(c(0, exp(-6), exp(-9)), c(0, 0, exp(-6)), c(0, 0, 0))
R> hmodel1 <- list(hmmNorm(mean = 100, sd = 16), hmmNorm(mean = 54, sd = 18),
+    hmmIdent(999))
R> fev1.msm <- msm(fev ~ days, subject = ptnum, data = fev,
+    qmatrix = three.q, hmodel = hmodel1, hcovariates = list(~ acute,
+    ~ acute, NULL), hconstraint = list(acute = c(1, 1)), death = 3,
+    method = "BFGS")
R> fev1.msm
R> sojourn.msm(fev1.msm)
  \end{CodeInput}
  \begin{CodeOutput}
Call:
msm(formula = fev ~ days, subject = ptnum, data = fev, qmatrix = three.q,
    hmodel = hmodel1, hcovariates = list(~acute, ~acute, NULL),
    hconstraint = list(acute = c(1,1)), death = 3, method = "BFGS")

Maximum likelihood estimates:
Transition intensity matrix

        State 1                            State 2
State 1 -0.0007038 (-0.0008333,-0.0005945) 0.0006275 (0.0005201,0.0007572)
State 2 0                                  -0.0008011 (-0.001013,-0.0006337)
State 3 0                                  0
        State 3
State 1 7.631e-05 (3.967e-05,0.0001468)
State 2 0.0008011 (0.0006337,0.001013)
State 3 0

Hidden Markov model, 3 states
Initial state occupancy probabilities:
        Estimate LCL UCL
State 1        1  NA  NA
State 2        0  NA  NA
State 3        0  NA  NA

State 1 - normal distribution
Parameters:
       Estimate       LCL       UCL
mean  98.004361 97.34297 98.665754
sd    16.185019 15.77782 16.602730
acute -8.791807 -9.95145 -7.632163

State 2 - normal distribution
Parameters:
       Estimate       LCL       UCL
mean  51.823341 50.76293 52.883748
sd    17.676307 17.08279 18.290443
acute -8.791807 -9.95145 -7.632163

State 3 - identity distribution
Parameters:
      Estimate LCL UCL
which      999  NA  NA

-2 * log-likelihood:  51597.89

R>  sojourn.msm(fev1.msm)
        estimates       SE         L        U
State 1  1420.759 122.3921 1200.0328 1682.084
State 2  1248.389 149.3041  987.5255 1578.161
  \end{CodeOutput}
\end{CodeChunk}

The estimated HMM normal outcome distributions show that in state 1,
FEV$_1$ measurements have a mean of 98\% baseline (SD 16\%) and in
state 2, a mean of 52\% baseline (SD 18\%).  FEV$_1$ is estimated to
be 9\% lower within 14 days of acute illnesses.  The function
\code{sojourn.msm} presents estimates and confidence intervals for $-1
/ q_{rr}$, indicating the average onset and progression rates of BOS
in days.  BOS state 1 is estimated to begin about 3 years (estimate
1420 days) after transplantation, and state 2 a further 3 years later.

The most likely true series of states underlying the data can be
estimated using the Viterbi algorithm~\citep{viterbi} through the
function \code{viterbi.msm}.  Figure~\ref{fig:fev} shows the most
likely time at which the individual passed from state 1 to state 2 --
the time when their decline below 80\% of baseline became sustained.


\subsection{Misclassification models}
\label{sec:misc}

An important special case of HMMs is the multi-state model with
misclassification, where the observed data are states, assumed to be
misclassifications of the true, underlying
states~\citep{jackson03:_multis_markov}.  In the CAV example, it is
not medically realistic for patients to recover from a diseased state
to a healthy state, as in the model of Section~\ref{sec:basic}.
Progression of coronary artery vasculopathy is thought to be an
irreversible process.  The angiography observations are actually
subject to error, which leads to some false measurements of CAV states
and apparent improvements in state.  Thus a more realistic Markov
intensity matrix $Q$ would be as given in Figure~\ref{fig:disease},
but with $q_{r+1,r} = 0$ for each $r$,
\[
Q = \left(
  \begin{array}{llll}
    -(q_{12} + q_{14}) & q_{12} &  0     & q_{14}\\
      0    & -(q_{23}+q_{24}) & q_{23} & q_{24}\\
      0    &   0    & -q_{34} & q_{34}\\
      0    &   0    &   0    &   0   \\
  \end{array}
\right ).
\]
We also assume that true state 1 (CAV-free) can be classified as state
1 or 2, state 2 (mild/moderate CAV) can be classified as state 1, 2 or
3, while state 3 (severe CAV) can be classified as state 2 or 3.
Recall that state 4 represents death.  Thus the matrix of
misclassification probabilities is
\[
E = \left(
  \begin{array}{llll}
    1 - e_{12} & e_{12} & 0 & 0 \\
    e_{21} & 1 - e_{21} - e_{23} & e_{23} & 0 \\
    0 & e_{32} & 1 - e_{32} & 0 \\
    0 & 0 & 0 & 0\\
  \end{array}
  \right)
\]
where $e_{rs}$ is the probability of observing state $s$ conditionally
on occupying true state $r$.

These are hidden Markov models with a categorical outcome
distribution, and as such may be fitted in \pkg{msm} using a
\code{hmmCat()} outcome distribution for each underlying state.
However \pkg{msm} provides a convenient shorthand for fitting models
of this form.  An \code{ematrix} argument to \code{msm()} is given a
matrix of initial values for the misclassification probabilities, with
zero in positions where misclassifications cannot occur.  In the CAV
example we initialize the four unknown misclassification parameters to
0.1, and set the initial values \code{oneway4.q} for $Q$ to the
approximate maximum likelihood estimates from the model without
misclassification.  \code{obstrue=firstobs} specifies that
observations indicated by the binary variable \code{firstobs} in the
data are not misclassifications, but observations of the true state.
In the CAV data, these are the dates of transplantation, at which
patients are known to be CAV-free, in state 1.

\begin{CodeChunk}
\begin{CodeInput}
R> ematrix <- rbind(c(0, 0.1, 0, 0), c(0.1, 0, 0.1, 0),
+    c(0, 0.1, 0, 0), c(0, 0, 0, 0))
R> oneway4.q <- rbind(c(0, 0.1, 0, 0.04), c(0, 0, 0.3, 0.05),
+    c(0, 0, 0, 0.3), c(0, 0, 0, 0))
R> rownames(oneway4.q) <- colnames(oneway4.q) <- c("Well", "Mild", "Severe",
+    "Death")
R> rownames(ematrix) <- colnames(ematrix) <- c("Well", "Mild", "Severe",
+    "Death")
R> misc.msm <- msm(state ~ years, subject = PTNUM, data = cav,
+    qmatrix = oneway4.q, ematrix = ematrix, obstrue = firstobs,
+    death = TRUE, method = "BFGS")
R> misc.msm
\end{CodeInput}

\pagebreak

\begin{CodeOutput}
Call:
msm(formula = state ~ years, subject = PTNUM, data = cav,
    qmatrix = oneway4.q, ematrix = ematrix, obstrue = firstobs, death = TRUE,
    method = "BFGS")

Maximum likelihood estimates:
Transition intensity matrix

       Well                      Mild
Well   -0.1317 (-0.1486,-0.1166) 0.0903 (0.07629,0.1069)
Mild   0                         -0.2917 (-0.3574,-0.238)
Severe 0                         0
Death  0                         0
       Severe                    Death
Well   0                         0.04136 (0.03318,0.05156)
Mild   0.2574 (0.1906,0.3475)    0.03429 (0.007473,0.1573)
Severe -0.3058 (-0.3878,-0.2412) 0.3058 (0.2412,0.3878)
Death  0                         0

Misclassification matrix

       Well                   Mild
Well   0.9726 (0.9539,0.9839) 0.02737 (0.01613,0.04605)
Mild   0.1751 (0.1007,0.2868) 0.7614 (0.61,0.8669)
Severe 0                      0.1143 (0.05691,0.2164)
Death  0                      0
       Severe                   Death
Well   0                        0
Mild   0.06353 (0.03667,0.1079) 0
Severe 0.8857 (0.7836,0.9431)   0
Death  0                        1 (1,1)

-2 * log-likelihood:  3910.098
\end{CodeOutput}
\end{CodeChunk}

Thus there is an estimated probability of about 0.03 that a patient
truly free of CAV will be diagnosed wrongly with mild CAV, but a
rather higher probability of 0.175 that underlying mild/moderate CAV
will be diagnosed as CAV-free.  Between the two CAV states, the mild
state will be misdiagnosed as severe with a probability of 0.06, and
the severe state will be misdiagnosed as mild with a probability of
0.11.  The model also estimates the progression rates through
underlying states.  An average of 8 years (1/0.1317) is spent
disease-free, an average of about 3 years is spent with mild/moderate
disease, and periods of severe disease also last about 3 years on
average before death.

The misclassification probabilities may also be modelled in terms of
covariates, using multinomial logistic regression.  This is
accomplished with the \code{misccovariates} argument to \code{msm()}.
For example, a disease screening test may be more sensitive for
different types of individuals.


\section{Model assessment}
\label{sec:assessment}

\citet{titman:review:jss} reviewed methods for assessing the fit of Markov
models to panel data. In particular, the Markov property and
homogeneity of transition rates, both between individuals and through
time, can be restrictive assumptions.


\subsection{Diagnostic plots}

One simple diagnostic compares model predictions of the entry time
into a particular state with nonparametric estimates, for example
Kaplan-Meier curves.  If the entry time is not observed exactly, then
the nonparametric estimate is an approximation.  In
Figure~\ref{fig:survfit}, the fit of four multi-state models to the
exactly-observed survival times in the CAV data is assessed in this
way.
\begin{CodeInput}
R> par(mfrow = c(2, 2))
R> plot.survfit.msm(cav.msm, main = "cav.msm: no covariates",
+    mark.time = FALSE)
R> plot.survfit.msm(cav.cov.msm, main = "cav.cov.msm: covariates",
+    mark.time = FALSE)
R> plot.survfit.msm(cav.pci.msm, mark.time = FALSE)
R> title("cav.pci.msm: time-inhomogeneous", line = 2)
R> title("(5 year change point)", line = 1)
R> cav.pci2.msm <- msm(state ~ years, subject = PTNUM, data = cav,
+    qmatrix = twoway4.q, death = 4, pci = c(5, 10), method = "BFGS",
+    control = list(maxit = 10000))
R> plot.survfit.msm(cav.pci2.msm, mark.time = FALSE)
R> title("cav.pci2.msm: time-inhomogeneous", line = 2)
R> title("(5, 10 year change points)", line = 1)
\end{CodeInput}
Up to about 10 years, all the models predict survival reasonably
accurately (within about 5\%).  The time-inhomogeneous model
\code{cav.pci.msm} fits slightly better than the time-homogeneous
models up to 10 years.  But the first three models overestimate
survival after 10 years -- 106 out of 614 individuals in the data live
beyond 10 years.  A further time-inhomogeneous model
\code{cav.pci2.msm} is fitted in which intensities change after 10 as
well as after 5 years, which substantially improves the fit both
before and after 10 years.

\begin{figure}[t!]
  \centering
  \scalebox{1.0}{\includegraphics{survfit}}
    \caption{Comparison of observed and fitted survival for three
    multi-state models for the CAV data.\label{fig:survfit}}
\end{figure}

Another common approach to multi-state model assessment is to compare
observed prevalences of states with expected prevalences under the
model at a series of times.  This can be done in \pkg{msm} using the
functions \code{prevalence.msm()} and \code{plot.prevalence.msm()}. To
compute observed prevalences precisely, all individuals should be
observed at these times.  If individuals are observed at different
times, this relies on approximations such as assuming transitions
occur only at observation times \citep{gentlaw} or at midpoints
between observation times.  Figure~\ref{fig:prev} presents a plot of
this type for the best-fitting model \code{cav.pci2.msm} for the CAV
data.  As time elapses, the proportions of individuals predicted to
have died appear to be underestimated by the model, and the
proportions alive and in states ``well'' and ``mild'' are
overestimated.  However, the Kaplan-Meier estimate in
Figure~\ref{fig:survfit} gives a more accurate estimate of the
``observed'' survival probability in this case.  The observed
prevalence of a state is simply calculated as the number of
individuals known to be in that state, divided by the number of
individuals whose state is known at that time, which ignores the
information from individuals censored at earlier times.


\subsection{Formal goodness-of-fit test}

\begin{figure}[t!]
  \centering
  \scalebox{1.0}{\includegraphics{prev}}
  \caption{Comparison of observed and expected prevalence from the time-inhomogeneous model \code{cav.pci2.msm} for the CAV data.\label{fig:prev}}
\end{figure}


The previous plots are informal diagnostics to suggest potential model
improvements.  A formal goodness-of-fit test for the hypothesis that
panel data were generated by a fitted Markov model was developed by
\citet{ahf}.  This test was extended by \citet{titman:sharples} to
handle exactly-observed death times and misclassified states.  This is
implemented in \pkg{msm} as the function \code{pearson.msm()}.  The
test compares observed and expected numbers of transitions between
pairs of states for a series of transition starting times, transition
time intervals and covariate categories, giving a Pearson-type
contingency table test statistic.

The null distribution of the statistic is not exactly $\chi^2$, with a
complex form for general panel data \citep{titman:asympnull}.  For
simpler models without covariates, \citet{ahf} showed by simulation
that the $\chi^2$ approximation was adequate.  The \code{pearson.msm}
function provides theoretical upper and (unless there are exact death
times) lower bounds for the test $p$~value.  In general cases, the null
distribution of the statistic can be estimated by the parametric
bootstrap procedure of repeatedly sampling from the fitted model,
refitting the model and recomputing the test statistic, resulting in
an accurate $p$~value.  If the resulting contingency table is sparse,
then the number of observation time, time interval or covariate
categories may need to be reduced to improve the $\chi^2$
approximation, though the power of the resulting test may be low.  See
the \code{pearson.msm} help page in the package for further details.

The Pearson-type test is performed for the four models illustrated in
Figure~\ref{fig:survfit}.  The upper $p$~value bounds indicate that none
of these models give an adequate overall fit.  This suggests that even
though the time-inhomogeneous model \code{cav.pci2.msm} fits well to
survival (Figure~\ref{fig:survfit}), it discriminates less well between
the states of CAV severity (Figure~\ref{fig:prev}).  A more complex
pattern of time-dependence, or allowing the transition intensities to
depend on covariates, would be expected to yield a better fit.

\begin{CodeChunk}
\begin{CodeInput}
R>  p1 <- pearson.msm(cav.msm)
R>  p1$test
\end{CodeInput}
\begin{CodeOutput}
    stat df.lower p.lower df.upper      p.upper
 165.047       NA      NA       81 1.072647e-07
\end{CodeOutput}
\begin{CodeInput}
R>  p2 <- pearson.msm(cav.cov.msm)
R>  p2$test
\end{CodeInput}
\begin{CodeOutput}
     stat df.lower p.lower df.upper     p.upper
 299.9516       NA      NA      241 0.005821544
\end{CodeOutput}
\begin{CodeInput}
R>  p3 <- pearson.msm(cav.pci.msm)
R>  p3$test
\end{CodeInput}
\begin{CodeOutput}
     stat df.lower p.lower df.upper      p.upper
 136.2905       NA      NA       81 0.0001188069
\end{CodeOutput}
\begin{CodeInput}
R>  p4 <- pearson.msm(cav.pci2.msm)
R>  p4$test
\end{CodeInput}
\begin{CodeOutput}
     stat df.lower p.lower df.upper     p.upper
 125.0847       NA      NA       81 0.001216962
\end{CodeOutput}
\end{CodeChunk}
Since the method of \citet{titman:sharples} to handle exactly-observed
death times involves multiple imputation of the next scheduled
observation time, these statistics and $p$~values include some simulation
error.  The default 100 imputations in this example ensures the
statistics have converged within 2 significant figures and the
$p$~values to within an order of magnitude.


\subsection{Other issues in model assessment}

The influence of each individual on the maximized likelihood can be
computed and illustrated by score residuals, using the function
\code{scoreresid.msm}.  \citet{titman:review:jss} also discussed the
assessment of multi-state models with misclassification, criticising
in particular the assumption of independence of the observed outcome
conditionally on the underlying state.


\section[Extensions of Markov models and limitations of msm]{Extensions of Markov models and limitations of \pkg{msm}}
\label{sec:limitations}

The \pkg{msm} package was designed to fit any Markov model structure
to panel-observed multi-state data.  Because of this aim of
generality, there are limitations in handling more complex models
which are only practicable for specific patterns of observations or
allowed transitions.
\subsection{Continuously-observed processes}
For example, if the data are continuously-observed, \pkg{msm} is
limited to exponential or piecewise-exponential sojourn times.  More
flexible models, for example, with Weibull-distributed sojourn times,
are relatively easy to fit to such data. The \pkg{mstate} package
\citep{mstate:jss,i07} implements multi-state models with nonparametric
baseline hazards and proportional hazards regression.

When the model is progressive, for example, a model as in Figure
\ref{fig:disease} but with all reverse transition rates $q_{r+1,r}=0$,
the number of possible pathways taken by an individual through the
states is finite, so that likelihood calculations are simpler.  For
example, the ``illness-death'' model has only one disease state, and
no recovery allowed from ``well'' to ``disease''.  The data for such a
model may only be \emph{interval-censored}, that is, the transition to
illness is known to have occurred between two observations, but at an
unknown time.  Flexible, non-parametric methods are possible in this
case \citep{frydman1995nem,frydman2008nem}.  This is simpler than
panel data, where both the type and number of transitions occurring
between adjacent observations are unknown in general.

\subsection{Time-inhomogeneous models}

Transition intensities may vary with time, depending on either the
time since the beginning of the process (a \emph{time-inhomogeneous}
model) or time since the previous transition (a \emph{semi-Markov}
model).  Time-inhomogeneous models in \pkg{msm} are restricted to
piecewise-constant intensities. The choice of change points is
unlimited, though in practice the results may be sensitive to this
choice.  Continuously-changing intensities, for example with a
Weibull-distributed time to the next transition, are generally more
scientifically plausible and may be more parsimonious.  The resulting
Kolmogorov differential equations for obtaining $P(u,t+u)$, hence the
likelihood for panel data, are analytically intractable, but can be
solved numerically in simpler instances.  For example,
\citet{hsiuhsichen2004smn} and \citet{hsieh2002acd} modelled only one
state with a time-varying sojourn distribution in this way.
\citet{hubbard2008mnm} fitted inhomogeneous models by estimating a
time transformation under which the inhomogeneous Markov model is
homogeneous, assuming the ratio of transition intensities stayed
constant through time.

\subsection{Non-Markov models}

Relaxing the Markov assumption with panel data presents more
difficulties.  Semi-Markov models with piecewise-constant intensities
are only feasible to estimate for simpler model structures
\citep{titman}.  \citet{foucher2008fsm:jss} used numerical integration to
compute the likelihood for 3 or 4 state progressive semi-Markov
models.  \citet{titman} described an Monte Carlo EM algorithm for
fitting progressive semi-Markov models to panel data.  All these
methods would be very difficult to implement for a general Markov
model structure.  In \pkg{msm}, an approximate non-Markov model might
be fitted by creating artificial time-dependent covariates
representing aspects of the process history, though this approach
would require very frequent observations to be sufficiently accurate.
A more promising approach to semi-Markov models is the
\emph{phase-type} model, in which the exponentially-distributed time
spent in each state $r$ is replaced by a series of exponential
sojourns (or ``phases'') in hidden states $r_1,\ldots,r_k$
\citep{titman:phasetype:jss}.  In principle, these models may be
implemented as hidden Markov models in \pkg{msm}, but certain
parameter constraints (currently not implemented) may be necessary for
identifiability.

\subsection{Random effects and Bayesian methods}

Unexplained heterogeneity in transition intensities between
individuals may be represented by random effects models, though these
are not implemented in \pkg{msm}.  Their likelihood for panel data is
intractable, except for specific cases such as the ``tracking'' model
\citep{satten1999eet} in which the random effect acts on all
intensities simultaneously, or a discrete random effects distribution
\citep{cook2004cmm}.

The \pkg{msm} package is limited to maximum likelihood estimation.
Multi-state models can be fitted to panel data from a Bayesian
perspective using MCMC simulation \citep{sharp:gibbs}, which is
particularly suited to hierarchical models with random effects.
Random effects Markov models with simple state structures have been
implemented using the \pkg{WinBUGS} \citep{winbugs} software for Bayesian
analysis \citep{pan2007markov,ardo:biostats:re}.
\citet{welton:ades:markov} describe how to implement general
multi-state structures using the \pkg{WBDiff} \citep{wbdiff} differential
equation solving interface to \pkg{WinBUGS} to calculate $P(t)$, while the
\pkg{JAGS} implementation of the \pkg{BUGS} language \citep{JAGS:proceedings} allows general
Markov model structures to be fitted to panel data via a distribution
\code{dmstate()}.


\subsection{Discrete-time models}

\pkg{msm} was designed for continuous-time models, but discrete-time
Markov and hidden Markov models can be fitted to discrete-time data
using \pkg{msm}, assuming that there is a continuous process
underlying the data.  The fitted transition probability matrix in one
time unit, $P(1)$, is then equivalent to the transition probability
matrix $P$ of the discrete-time model.  But since a discrete-time
Markov model is equivalent to a series of multinomial models for each
observation conditionally on the previous observation, these may be
fitted more efficiently using software for multinomial logistic
regression, for example, the function \code{multinom()} in the
\proglang{R} package \pkg{nnet}~\citep{venables:ripley}.  Currently
there are several available \proglang{R} packages which can fit
discrete-time hidden Markov models of various forms, for example
\pkg{HiddenMarkov} \citep{HiddenMarkov}, \pkg{hsmm} \citep{hsmm} and
\pkg{mhsmm} \citep{mhsmm}.

\section{Further information}

This article gives an overview of the \pkg{msm} package for fitting
continuous-time Markov and hidden Markov models to panel data.
Detailed references for all the functions for model fitting and output
presentation are available as help pages in the installed package.
The \code{doc} subdirectory of the package also contains a user guide
in PDF format, which presents much of the material in this article in
greater detail.

The examples in this article were run using version 1.0 of \pkg{msm},
available from \url{http://CRAN.R-project.org/package=msm}.

\section*{Acknowledgments}

Many thanks to Linda Sharples and other colleagues at the MRC
Biostatistics Unit for encouraging the development of the \pkg{msm}
package, to Andrew Titman for his thorough work on model assessment
and elaborated models, and to Martyn Plummer for contributing code for
matrix exponentiation.  Thanks also to the many users of \pkg{msm} for
their comments, encouragement and bug reports.  The author is funded
by the UK Medical Research Council (grant U.1052.00.008).

\bibliography{v38i08}

\end{document}
